%%%%%%%%%%%%%%%%%%%%%%%%%%%%%%%%%%%%%%%%%%%%%%%%%%%%%%%%%%%%%%%%%%
%%%                      Homework 1                            %%%
%%%%%%%%%%%%%%%%%%%%%%%%%%%%%%%%%%%%%%%%%%%%%%%%%%%%%%%%%%%%%%%%%%

\documentclass[letter]{article}

\usepackage{lipsum}
\usepackage[pdftex]{graphicx}
\usepackage[margin=1.5in]{geometry}
\usepackage[english]{babel}
\usepackage{listings}
\usepackage{amsthm}
\usepackage{amssymb}
\usepackage{framed} 
\usepackage{amsmath}
\usepackage{titling}
\usepackage{fancyhdr}
\usepackage{enumerate}

%%%%%%%%%%%%%%%
%% DOC INFO %%%
%%%%%%%%%%%%%%%
\newcommand{\bHWN}{2}
\newcommand{\bCLASS}{CS 189}

\title{\bCLASS: Homework \bHWN}
\author{William Guss\\26793499\\wguss@berkeley.edu}

\fancyhead[L]{\bCLASS}
\fancyhead[CO]{Homework \bHWN}
\fancyhead[CE]{GUSS}
\fancyhead[R]{\thepage}
\fancyfoot[LR]{}
\fancyfoot[C]{}
\usepackage{csquotes}




\pagestyle{fancy}


\newtheorem{theorem}{Theorem}
\newtheorem{definition}{Definition}
\newtheorem{lemma}{Lemma}

\newenvironment{menumerate}{%
  \edef\backupindent{\the\parindent}%
  \enumerate%
  \setlength{\parindent}{\backupindent}%
}{\endenumerate}

%%%%%%%%%%%%%%

\begin{document}
\maketitle
\thispagestyle{empty}

\begin{menumerate}
\item 
\item
\item Let $A$ be a positive definite matrix in $\mathbb{R}^{n\times n}.$
  \begin{menumerate}
    \item Cionsider the following derrivation:
    \begin{equation}
      \begin{aligned}
          x^TAx = x^T \begin{bmatrix}
            \sum_j^n  a_{1j}x_j \\
            \vdots\\
            \sum_j^n a_{nj}x_j
          \end{bmatrix}  = \sum_i^n \sum_j^n a_{ij}x_ix_j.
       \end{aligned}   
    \end{equation}
    \item 
    \begin{theorem}
      If $A$ is positive definite, then the diagonals of $A$ are positive.
    \end{theorem}
    \begin{proof}
      Suppose that there is negative value on the diagonal, say $a_{qq}.$
      Then let $x = e_{q}.$ If we apply the quadratic form we get $e_qA^Te_q = a_qq < 0.$ 
      This contradicts the positive semidefiniteness of $A.$
    \end{proof}
  \end{menumerate}
  \item Short Proofs. 
  \begin{menumerate}
    \item Assume problem (b).
    \begin{lemma}
      If $A$ is a matrix with eigen values $\lambda_n$  $A + \gamma I$ has eigenvalues $\gamma + \lambda_n$   
    \end{lemma}
    \begin{proof}
      If $\lambda_n$ is an eigenvalue, then $Av_n = \lambda_n v_n$  for a corresponding eigenvector $v.$
      Furthermore 
      \begin{equation}
       (A + \gamma I)v = Av + \gamma I v = \lambda_n v + \gamma v = (\lambda_n + \gamma)
       \end{equation} 
       which implies that $\lambda_n + \gamma$ is an eigen value of $A + \gamma I.$ This completes the proof.  
    \end{proof}
    \begin{theorem}
      If $A$ is positive semidefinite and $\gamma > 0,$ then $A + \gamma I$ is positive definite.
    \end{theorem}
    \begin{proof}
      If $A$ is positive definite then by the logic of the proof of (b), 
      \begin{equation}
          \begin{aligned}
            x^T A x = \sum_i \lambda_i (x_i^T e_i)^2 \geq 0.         
          \end{aligned}
         \end{equation}   
      It follows that some $\lambda \geq 0$ since $x \neq 0.$ Therefore by the previous lemma adding 
      $\gamma$ to the diagonal adds $\gamma$ to every eigenvalue implying that all eigen values are positive.
      By (b), $A + I\gamma$ is positive definiute therefore.
      \end{proof}
    \item Lolololol!
    \begin{theorem}
      $A$ is positive definite if and only if all of its eigen values are more than 0.
    \end{theorem}
    \begin{proof}
      Iff $A$ is positive semidefinite then it is symmetric. Using spectral theorem we have that 
      \begin{equation}
        \begin{aligned}
          x^TAx = \sum_i (x^Te_i)e_i^TAx &= \sum_i = x^Te_ie_i^T\lambda e_i^Tx \\
              &= \sum_i \lambda_i (x^Te_i)^2 > 0
         \end{aligned}   
      \end{equation}   
      which is true if and only if all $\lambda_i$ are more than 0.
    \end{proof}
    \item 
    \begin{theorem}
      If $A$ is positive definite then it is invertible.   
    \end{theorem}
    \begin{proof}
      The invertible matrix theorem statesa that a matrix is invertible if and only if all of its eigen values
      are more than $0$. By the previous theorem if $A$ is positive definite then all of its eigen values
       are positive and so it is invertible.   
    \end{proof}
    \item
    \begin{theorem}
      If $A$ is positive definite then there exist $n$ linearly independent vectors so that 
      $A_{ij} = x^T_ix_j.$   
    \end{theorem}
    \begin{proof}
      The statement of the theorem is true if and only if $A = B^TB$ where $B$ is invertible. 
      By spectral theorem we have that $A = U\Lambda U^T$ where $\Lambda = diag(\lambda_1, \dots, \lambda_n).$
      Furthermore $U^{-1} = U^T$. Let $\Omega = diag(\sqrt \lambda_1, \dots, \sqrt \lambda_n).$ Then, 
      $\Omega^2 = \Lambda.$ Let $W^T = U\Omega$ and $W = \Omega U^T.$ So we have that $W$ is still an orthonormal matrix
      and so $A = W^TW.$ This completes the proof.
    \end{proof}
  \end{menumerate}
  \item DERIVATIONS :( Assuming theorems from Math 105
  \begin{menumerate}
      \item Consider the followiong derivation
      \begin{equation}
        \begin{aligned}
          \frac{\partial(x^Ta)}{\partial x} = \frac{\partial (x)}{\partial x}^T a + \left(\frac{\partial (a)}{\partial x}\right)^T x
          &= a.
        \end{aligned}
       \end{equation}
       \item Consider the following derivation
       \begin{equation}
          \begin{aligned}
              \frac{\partial(x^TAx)}{\partial x} = \frac{\partial(x^T)}{\partial x} Ax + \frac{\partial(Ax)}{\partial x}^TX = Ax + Ax^T
          \end{aligned}
       \end{equation}
       \item Consider the following derivation
       \item
       \begin{theorem}
          If $x \in \mathbb{R}^n$ 
          \begin{equation}
            \|x\|_2 \leq \|x\|_1 \leq \sqrt{n}\|x\|_2 = \sqrt{\sum_{i=1}^n x_i^2}.
          \end{equation}
       \end{theorem}
       \begin{proof}
            
       \end{proof}
  \end{menumerate}
\end{menumerate}

\end{document}





